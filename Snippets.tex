% Generated by generate_pdf.py
\documentclass[10pt,landscape,twocolumn,a4paper,notitlepage]{article}
\usepackage{hyperref}
\usepackage[english, activeacute]{babel}
\usepackage[utf8]{inputenc}
\usepackage[T1]{fontenc}
\usepackage{lmodern}
\usepackage{fancyhdr}
\usepackage{lastpage}
\usepackage{listings}
\usepackage{listingsutf8}
\usepackage{listingsutf8}
\usepackage{amssymb}
\usepackage[usenames,dvipsnames]{color}
\usepackage{graphicx}
\usepackage{wrapfig}
\usepackage{amsmath}
\usepackage{makeidx}

%%% Margenes
\setlength{\columnsep}{0.25in}
\setlength{\columnseprule}{0.5pt}

\addtolength{\textheight}{2.35in}
\addtolength{\topmargin}{-0.9in}

\addtolength{\textwidth}{1.1in}
\addtolength{\oddsidemargin}{-0.55in}

\setlength{\headsep}{0.08in}
\setlength{\parskip}{0in}
\setlength{\headheight}{15pt}
\setlength{\parindent}{0mm}

%%% Encabezado y pie de pagina
\pagestyle{fancy}
\fancyhead[LO]{\textbf{\title}}
\fancyhead[C]{\leftmark\ -\ \rightmark}
\fancyhead[RO]{Page \thepage\ of \pageref{LastPage}}
\renewcommand{\headrulewidth}{0.4pt}
\fancyfoot{}
\definecolor{darkblue}{rgb}{0,0,0.4}
%%% Configuracion de Listings
\lstloadlanguages{C++}
\lstnewenvironment{code}
    {\csname lst@SetFirstLabel\endcsname}
    {\csname lst@SaveFirstLabel\endcsname}
\lstset{% general command to set parameter(s)
    language=C++, basicstyle=\small\ttfamily, keywordstyle=\slshape,
    emph=[1]{tipo,usa}, emphstyle={[1]\sffamily\bfseries},
    morekeywords={tint,forn,forsn,fore},
    basewidth={0.47em,0.40em},
    columns=fixed, fontadjust, resetmargins, xrightmargin=5pt, xleftmargin=15pt,
    flexiblecolumns=false, tabsize=2, breaklines, breakatwhitespace=false, extendedchars=true,
    inputencoding=utf8,
    numbers=left, numberstyle=\tiny, stepnumber=1, numbersep=9pt,
    frame=l, framesep=3pt,
    basicstyle=\ttfamily,
    keywordstyle=\color{darkblue}\ttfamily,
    stringstyle=\color{magenta}\ttfamily,
    commentstyle=\color{RedOrange}\ttfamily,
    morecomment=[l][\color{OliveGreen}]{\#},
    literate={á}{{\'a}}1 {é}{{\'e}}1 {í}{{\'\i}}1 {ó}{{\'o}}1 {ú}{{\'u}}1
             {Á}{{\'A}}1 {É}{{\'E}}1 {Í}{{\'I}}1 {Ó}{{\'O}}1 {Ú}{{\'U}}1
             {ñ}{{\~n}}1 {Ñ}{{\~N}}1 {ü}{{\"u}}1 {Ü}{{\"U}}1
}

\lstdefinestyle{C++}{
    language=C++, basicstyle=\small\ttfamily, keywordstyle=\slshape,
    emph=[1]{tipo,usa,tipo2}, emphstyle={[1]\sffamily\bfseries},
    morekeywords={tint,forn,forsn,fore},
    basewidth={0.47em,0.40em},
    columns=fixed, fontadjust, resetmargins, xrightmargin=5pt, xleftmargin=15pt,
    flexiblecolumns=false, tabsize=2, breaklines, breakatwhitespace=false, extendedchars=true,
    numbers=left, numberstyle=\tiny, stepnumber=1, numbersep=9pt,
    frame=l, framesep=3pt,
    basicstyle=\ttfamily,
    keywordstyle=\color{darkblue}\ttfamily,
    stringstyle=\color{magenta}\ttfamily,
    commentstyle=\color{RedOrange}\ttfamily,
    morecomment=[l][\color{OliveGreen}]{\#}
}

%%% Macros
\def\nbtitle#1{\begin{Large}\begin{center}\textbf{#1}\end{center}\end{Large}}
\def\nbsection#1{\section{#1}}
\def\nbsubsection#1{\subsection{#1}}
\def\nbcoment#1{\begin{small}\textbf{#1}\end{small}}
\newcommand{\comb}[2]{\left( \begin{array}{c} #1 \\ #2 \end{array}\right)}
\def\complexity#1{\texorpdfstring{$\mathcal{O}(#1)$}{O(#1)}}
\newcommand\cppfile[2][]{
    \lstinputlisting[style=C++,#1]{\detokenize{#2}}
}

\begin{document}
\def\title{El Bicho}
\vspace{0.6cm}
\centering{\LARGE\textbf{El Bicho}}\\[0.5cm]
\centering{DondeEstasCR7}\\[0.5cm]
\centering{\includegraphics[width=5.5cm]{img/cr7.jpg}}\\[0.5cm]
\centering{18/08/2025}\\[0.2cm]
\tableofcontents
\newpage

\section{Algos}
\subsection{Fast Io}
\cppfile{build/sanitized_include/Fast IO.cpp}

\section{Bit Manipulation}
\textit{Técnicas para manipular bits individuales y operaciones a nivel de bit. Incluye macros útiles para competencias de programación.}\
\subsection{Bits}
\textit{Macros esenciales para manipulación de bits: verificar potencias de 2, establecer/limpiar bits, contar bits, y operaciones con LSB/MSB.}\
\cppfile{build/sanitized_include/Bits.cpp}

\section{Combinatory}
\subsection{Combi Brute Sin Mod}
\cppfile{build/sanitized_include/Combi_brute_sin_MOD.cpp}
\subsection{Combinatory}
\textit{OJO: Es necesario usar binpow con MOD primo}\
\cppfile{build/sanitized_include/Combinatory.cpp}

\section{Graph}
\textit{Algoritmos de grafos: DFS, BFS, componentes fuertemente conexas, y otras estructuras de datos para problemas de grafos.}\
\subsection{Bfs}
\cppfile{build/sanitized_include/BFS.cpp}
\subsection{Bipartite}
\cppfile{build/sanitized_include/Bipartite.cpp}
\subsection{Dfs}
\cppfile{build/sanitized_include/DFS.cpp}
\subsection{Dfs 2D}
\cppfile{build/sanitized_include/DFS_2D.cpp}
\subsection{Disjoint Set Union Dsu}
\cppfile{build/sanitized_include/Disjoint Set Union DSU.cpp}
\subsection{Djisktra}
\cppfile{build/sanitized_include/Djisktra.cpp}
\subsection{Lowest Common Ancestor Lca}
\cppfile{build/sanitized_include/Lowest Common Ancestor LCA.cpp}
\subsection{Scc}
\textit{Algoritmo de Tarjan para encontrar componentes fuertemente conexas (SCC) en un grafo dirigido.}\
\cppfile{build/sanitized_include/SCC.cpp}
\subsection{Topological Sort}
\cppfile{build/sanitized_include/Topological Sort.cpp}

\section{Number Theory}
\textit{Primeros 180 Primos:
2, 3, 5, 7, 11, 13, 17, 19, 23, 29, 31, 37, 41, 43, 47, 53, 59, 61, 67, 71, 73, 79, 83, 89, 97, 101, 103, 107, 109, 113, 127, 131, 137, 139, 149, 151, 157, 163, 167, 173, 179, 181, 191, 193, 197, 199, 211, 223, 227, 229, 233, 239, 241, 251, 257, 263, 269, 271, 277, 281, 283, 293, 307, 311, 313, 317, 331, 337, 347, 349, 353, 359, 367, 373, 379, 383, 389, 397, 401, 409, 419, 421, 431, 433, 439, 443, 449, 457, 461, 463, 467, 479, 487, 491, 499, 503, 509, 521, 523, 541, 547, 557, 563, 569, 571, 577, 587, 593, 599, 601, 607, 613, 617, 619, 631, 641, 643, 647, 653, 659, 661, 673, 677, 683, 691, 701, 709, 719, 727, 733, 739, 743, 751, 757, 761, 769, 773, 787, 797, 809, 811, 821, 823, 827, 829, 839, 853, 857, 859, 863, 877, 881, 883, 887, 907, 911, 919, 929, 937, 941, 947, 953, 967, 971, 977, 983, 991, 997, 1009, 1013, 1019, 1021, 1031, 1033, 1039, 1049, 1051, 1061, 1063, 1069.}\
\subsection{Euler Toliente}
\cppfile{build/sanitized_include/Euler_Toliente.cpp}
\subsection{Gcd Lcm}
\cppfile{build/sanitized_include/GCD_LCM.cpp}
\subsection{Number Theory}
\cppfile{build/sanitized_include/Number_Theory.cpp}
\subsection{Phi Euler}
\textit{Phi(n) = contar la cantidad de numero coprimos entre 1 a n}\
\cppfile{build/sanitized_include/Phi_Euler.cpp}
\subsection{Potenciacion Binaria}
\cppfile{build/sanitized_include/Potenciacion_Binaria.cpp}
\subsection{Sieve}
\cppfile{build/sanitized_include/Sieve.cpp}
\subsection{Sieve Bitset}
\cppfile{build/sanitized_include/Sieve_bitset.cpp}
\subsection{Sum Of Divisors}
\cppfile{build/sanitized_include/Sum_of_Divisors.cpp}

\section{Segment Tree}
\subsection{Find Two Numbers}
\cppfile{build/sanitized_include/Find_two_Numbers.cpp}
\subsection{Segment Tree Recursivo}
\cppfile{build/sanitized_include/Segment_Tree_Recursivo.cpp}
\subsection{Segment Tree V2}
\cppfile{build/sanitized_include/Segment_Tree_v2.cpp}
\subsection{Segment Tree V3}
\cppfile{build/sanitized_include/Segment_Tree_v3.cpp}

\end{document}
